\documentclass[10pt,a4paper,sans]{moderncv} 
\usepackage{setspace}
\moderncvstyle{classic} 
\moderncvcolor{blue} 

\usepackage[scale=0.90]{geometry}

\name{Ricardo Erick}{Torres Rosas}
\title{Ingeniero en Sistemas Computacionales} 
\address{Av. Fuentes Brotantes,}{}{C.P.14410, Tlalpan, CDMX}
\phone[mobile]{+52~(55)~44~36~69~71} 
\email{r.erick.torres.r@mail.com} 
\social[linkedin]{ricardo-etr}
\social[github]{ricardoETR}
\photo[70pt][0.4pt]{portrait}
\begin{document}
\setstretch{1.15}
\makecvtitle
\section{Perfil Profesional}
\cvitem{}{Soy un profesional proactivo con más de 2 años de experiencia en análisis y desarrollo de software. Destaco por mi capacidad para resolver problemas y crear soluciones innovadoras, como la optimización de Stored Procedures, reduciendo su tiempo de ejecución en un 30\%. Además de habilidades técnicas, poseo habilidades blandas excepcionales, incluyendo trabajo en equipo, comunicación efectiva y adaptabilidad. Mi enfoque se centra en aplicar mis conocimientos para contribuir al éxito y crecimiento del equipo. Busco oportunidades para seguir aprendiendo y creciendo profesionalmente, contribuyendo al cumplimiento de objetivos organizacionales.} 
\section{Experiencia}
\cventry{Marzo 2022 - Actual}{Analista en Sistemas}{Inbursa}{México}{}{
    \begin{itemize}
        \item Análisis detallado de requerimientos para comprender las necesidades del proyecto y elaboración de Project Charters.
        \item Desarrollo de servicios RESTful para diversas funciones como envío de correos electrónicos, encriptación de archivos PDF y descarga de datos en formato .xlsx.
        \item Creación, optimización y mantenimiento de bases de datos en SQL Server, abarcando gestión de tablas, Stored Procedures, Triggers e Index.
        \item Mantenimiento y mejora continua del aplicativo de Estados de Cuenta en C\# .NET.
        \item Diseño, optimización y mantenimiento de flujos de Automation y Designer.
        \item Implementación de flujos en Pentaho para la generación de informes estadísticos sobre productos y subproductos de los Estados de Cuenta.
        \item Realización de pruebas exhaustivas, incluyendo pruebas de escritorio, pruebas unitarias y pruebas en colaboración con auditores, para validar resultados y garantizar integridad y precisión de los sistemas desarrollados.
    \end{itemize}   
}
\vspace{0.5cm}
\cventry{Enero 2021 - Enero 2022}{Prácticas Profesionales}{Escuela Superior de Cómputo}{IPN}{}{
    \begin{itemize}
        \item Miembro de proyecto para 
        la creación de aplicación móvil con Realidad Aumentada para visita al Museo Soumaya.
        Uso de Tecnologías como C\# Unity 3D, Vuforia RA, Blender, Sqlite, Git, entre otros.
    \end{itemize}    
}

\section{Educación}
\cventry{Julio 2016 - Enero 2022}{Ingeniero en Sistemas Computacionales}{Escuela Superior de Cómputo}{México}{}{}  

\section{Habilidades} 
\cvitem{Lenguajes}{C\#, Python, C++ SQL, LaTeX, HTML, CSS, JavaScript}
\cvitem{Herramientas}{Git, GitHub, Pentaho, Visual Studio, Jira, Scrum, Kanban}

\section{Certificaciones}
\cventry{2023}{Python: Comprehensions, Funciones y manejo de errores}{Platzi}{México}{}{}
\cventry{2023}{Algoritmos y diagramas de flujo}{Platzi}{México}{}{}
\cventry{2023}{Ética y manejo de datos para data Science e inteligencia artificial}{Platzi}{México}{}{}
\cventry{2023}{Introdución a Excel}{Platzi}{México}{}{}
\cventry{2022}{Scrum Fundamentals Certified}{SCRUMstudy}{México}{}{}

\section{Idiomas}
% \cvitem{Español}{Nativo}
\cvitem{Inglés}{B2}
\end{document}